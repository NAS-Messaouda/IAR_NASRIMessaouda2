\documentclass[14 pt]{article}
\usepackage[utf8]{inputenc}
\title{Wireless Sensor Network clustering}
\usepackage{graphicx}
\graphicspath{ {images/} }
\begin{document}
Adrar  University  Faculty of Science and Technology 
Department of Mathematics and Computer Science 
    (Initiation to Research 1) 2020/2021 
\section*{ Title of  the article }
Wireless Sensor Network clustering
\section*{Introduction}
Is an emerging specialized technology with communications infrastructure to monitor and record conditions in different locations, made from tens to thousands of sensor nodes that spread within the phenomenon, working to capture and sense of the type of phenomena .
 \begin{figure}[h]
    \centering
    \includegraphics[width=0.75 \textwidth]{WSN}
    \caption{model of cluster-based WSN }
    \label{fig:WSN}
\end{figure} 
 
\section*{Aim}
 Is Data aggregation , by reducing the amount of excess information transmitted by sensors, consumes less power in the network and thus improves its life. 
\section*{Arguments}
  - In order to reach each side in the search area, the signal sent to the cluster head in turn sends it to the base station. \linebreak
  - This type of network allows us to study the surroundings in the event of phenomena: fire, heat of the area ... ect. 
\section*{Objectives}
 The initial ideas for how to achieve the goal is to study cluster types and we either merge two clusters to get a new cluster, or discover another species through 
 in-depth study.
 \linebreak
\section*{Information about the writer}
Full name : NASRI Messaouda    
\linebreak Contact information : (nas.messaouda@gmail.com)
 \linebreak 
 \linebreak Nots : is just initial draft of my project proposal.
\end{document}
